%%
%% The "title" command has an optional parameter,
%% allowing the author to define a "short title" to be used in page headers.
%\title[Principled XAI]{The Rise of Principled XAI for Trustworthy AI}
%\title[Principled XAI]{Trustworthy AI by Principled XAI}
\title[Symbolic Explainable AI]{Symbolic Explainable Artificial Intelligence}

%%
%% The "author" command and its associated commands are used to define
%% the authors and their affiliations.
%% Of note is the shared affiliation of the first two authors, and the
%% "authornote" and "authornotemark" commands
%% used to denote shared contribution to the research.

\author{Ramon B\'{e}jar}
\email{ToDo}
\orcid{}

\affiliation{%
  \institution{\\
    University of Lleida}
  \city{Lleida}
  \country{Spain}
  \postcode{}
}

\author{Antonio Morgado}
\email{ToDo}
\orcid{}

\affiliation{%
  \institution{\\
    University of Lisbon, IST/INESC-ID}
  \city{Lisbon}
  \country{Portugal}
  \postcode{}
}

\author{Jordi Planes}
\email{ToDo}
\orcid{}

\affiliation{%
  \institution{\\
    University of Lleida}
  \city{Lleida}
  \country{Spain}
  \postcode{}
}

\author{Joao Marques-Silva}
\email{jpms@acm.org}
\orcid{0000-0002-6632-3086}

\affiliation{%
  \institution{\\
    ICREA, University of Lleida}
  %\streetaddress{P.O. Box 1212}
  \city{Lleida}
  %\state{Ohio}
  \country{Spain}
  \postcode{}
}

%%
%% By default, the full list of authors will be used in the page
%% headers. Often, this list is too long, and will overlap
%% other information printed in the page headers. This command allows
%% the author to define a more concise list
%% of authors' names for this purpose.
\renewcommand{\shortauthors}{Bejar, Morgado, Planes, Marques-Silva}

%%
%% The code below is generated by the tool at http://dl.acm.org/ccs.cfm.
%% Please copy and paste the code instead of the example below.
%%


\begin{CCSXML}
<ccs2012>
<concept>
<concept_id>10003752.10003790.10003794</concept_id>
<concept_desc>Theory of computation~Automated reasoning</concept_desc>
<concept_significance>500</concept_significance>
</concept>
<concept>
<concept_id>10003752.10003790.10003795</concept_id>
<concept_desc>Theory of computation~Constraint and logic programming</concept_desc>
<concept_significance>500</concept_significance>
</concept>
<concept>
<concept_id>10003752.10003790.10002990</concept_id>
<concept_desc>Theory of computation~Logic and verification</concept_desc>
<concept_significance>500</concept_significance>
</concept>
</ccs2012>
\end{CCSXML}

\ccsdesc[500]{Theory of computation~Automated reasoning}
\ccsdesc[500]{Theory of computation~Constraint and logic programming}
\ccsdesc[500]{Theory of computation~Logic and verification}

%%
%% Keywords. The author(s) should pick words that accurately describe
%% the work being presented. Separate the keywords with commas.
\keywords{Explainability, Symbolic AI, Logic-Based Explanations}

%% A "teaser" image appears between the author and affiliation
%% information and the body of the document, and typically spans the
%% page.
%\begin{teaserfigure}
%  \includegraphics[width=\textwidth]{sampleteaser}
%  \caption{Seattle Mariners at Spring Training, 2010.}
%  \Description{Enjoying the baseball game from the third-base
%  seats. Ichiro Suzuki preparing to bat.}
%  \label{fig:teaser}
%\end{teaserfigure}


%%
%% The abstract is a short summary of the work to be presented in the
%% article.
\begin{abstract}
  With the rapid advances in artificial intelligence (AI) in recent
  years, it has become critical to devise solutions to achieve
  trustworthy AI. One such solution is to enable machine learning (ML)
  models, and systems of artificial intelligence (AI) in general, with
  the ability to explain their operation. This area of research is
  generally referred to as eXplainable AI (XAI), and it has been the
  subject of massive interest in recent years.
  %
  However, most works on XAI are based on methods of non-symbolic AI,
  and exhibit critical shortcomings. These shortcomings can be shown
  to represent a variety of sources of error. Evidently, a credible
  foundation of trustworthy AI cannot be based on methods that can and
  often do produce erroneous results.
  %
  It is therefore apparent that most existing methods of XAI are
  unsuitable as a foundation for trustworthy AI.
  %
  The natural alternative to approaches of XAI based on non-symbolic
  AI are methods of symbolic AI, aiming at delivering rigorous
  solutions of XAI.
  %
  The purpose of this paper is to provide an in-depth overview of the
  progress that has been observed in the in the last few
  years in the emerging field of symbolic XAI, but also to identify
  existing research challenges.
\end{abstract}


%%
%% This command processes the author and affiliation and title
%% information and builds the first part of the formatted document.
\maketitle
